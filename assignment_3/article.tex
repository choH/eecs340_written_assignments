\documentclass[11pt]{article}
\usepackage{setspace}
\setstretch{1}
\usepackage{amsmath,amssymb, amsthm}
\usepackage{graphicx}
\usepackage{bm}
\usepackage[hang, flushmargin]{footmisc}
\usepackage[colorlinks=true]{hyperref}
\usepackage[nameinlink]{cleveref}
\usepackage{footnotebackref}
\usepackage{url}
\usepackage{listings}
\usepackage[most]{tcolorbox}
\usepackage{inconsolata}
\usepackage[papersize={8.5in,11in}, margin=1in]{geometry}
\usepackage{float}
\usepackage{caption}
\usepackage{esint}
\usepackage{url}
\usepackage{enumitem}
\usepackage{subfig}
\usepackage{wasysym}
\newcommand{\inlinecode}{\texttt}
\usepackage{etoolbox}
\usepackage{algorithm}
% \usepackage{algorithmic}
\usepackage[noend]{algpseudocode}
\usepackage{tikz}
\usetikzlibrary{matrix,positioning,arrows.meta,arrows}
\patchcmd{\thebibliography}{\section*{\refname}}{}{}{}

\makeatletter
\renewcommand{\@seccntformat}[1]{}
\makeatother


\begin{document}



\title{\textbf{EECS 340: Assignment 3}}

\author{Shaochen (Henry) ZHONG, \inlinecode{sxz517} \\ Yuhui ZHANG, \inlinecode{yxz2052}}
\date{Due and submitted on 02/20/2020 \\ EECS 340, Dr. Koyut{\"u}rk}
\maketitle

% % % % % % % % % % % % % % % % % % % % % % % % % % % % % % % % % %
% % % % % % % % % % % % % % % % % % % % % % % % % % % % % % % % % %
% % % % % % % % % % % % % % % % % % % % % % % % % % % % % % % % % %
\section{Problem 1}

\section{(a) $T(n) = bT(n/a) + \Theta(n)$}

For this recurrance, we have a comparsion between $n^{\log_a b}$ and $n$. Since it is given that $1 < a < b$, there must be $\log_a b > 1$. Therefore we may say that there must be a $n^\epsilon = \frac{n^{\log_a b}}{n}$ where $0 < \epsilon = \log_a b - 1$.

Now we have $f(n) = n = O(n^{(\log_a b) - \epsilon})$, where $0 < \epsilon = \log_a b - 1$, we can apply \textit{case 1} of the master theorem and conclude that the solution is $T(n) = \Theta(n)$.

\section{(b) $T(n) = a^2 T(n/a) + \Theta(n^2)$}
For this recurrance, we have a comparsion between $n^{\log_a a^2}$ and $n^2$, thus $n^2$ and $n^2$. As now we have $f(n) = \Theta(n^2)$, we can apply \textit{case 2} of the master theorem and conclude that the solution is $T(n) = \Theta(n \cdot \log n)$


\section{(c) $T(n) = T(\lambda n) + n^{\lambda}$}

We may rewrite it as $T(n) = T\Big(\frac{n}{\frac{1}{\lambda}}\Big) + n^{\lambda}$. Thus, we have a comparsion between $n^{\log_{\frac{1}{\lambda}} 1}$ and $n^{\lambda}$, which is equivalent as $n^0$ and $n^{\lambda}$, then we have $f(n) = \Omega(n^{\log_b a + \epsilon})$ for $\epsilon = \lambda$. We may also show $af(\frac{n}{b}) \leq cf(n)$ for $1 \cdot f\Big(\frac{n}{\frac{1}{\lambda}}\Big) = f(\lambda n)$. Combined, together, we can apply \textit{case 3} of the master theorem and conclude that the solution is $T(n) = \Theta(n^\lambda)$.

\section{(d) $T(n) = aT(\frac{n}{a}) + \Theta(n^{\lambda}(\log n)^b)$}
For this recurrance, we have a comparsion between $n^{\log_a a}$ and $n^{\lambda}(\log n)^b)$, which is equivalent to comparing $n$ and $n^{\lambda}(\log n)^b)$. We may prove that $n$ is polynomially larger than $n^{\lambda}(\log n)^b)$ by analyzing:

\begin{align}
    \text{W.T.S. \ }\lim_{n \to \infty} \frac{n^\epsilon \cdot n^\lambda (\log n)^b}{n} &= 0\\
    \lim_{n \to \infty} \frac{n^\lambda (\log n)^b}{n^{1-\epsilon - \lambda}} &= 0 \nonumber\\
    % \text{Since $n^x$ for any $x > 0$ grows much faster than anything with a base of $\log n$, we must have the following} \nonumber \\
    \Longrightarrow 1-\epsilon - \lambda > 0 &\Rightarrow \epsilon < 1 - \lambda
\end{align}

Thus, we have $f(n) = O(n^{\log_b a - \epsilon})$ for $\epsilon < 1 - \lambda$. Then, we can apply \textit{case 1} of the master theorem and conclude that the solution is $T(n) = \Theta(n)$.




% % % % % % % % % % % % % % % % % % % % % % % % % % % % % % % % % %
% % % % % % % % % % % % % % % % % % % % % % % % % % % % % % % % % %
% % % % % % % % % % % % % % % % % % % % % % % % % % % % % % % % % %
\section{Problem 2}


% % % % % % % % % % % % % % % % % % % % % % % % % % % % % % % % % %
% % % % % % % % % % % % % % % % % % % % % % % % % % % % % % % % % %
% % % % % % % % % % % % % % % % % % % % % % % % % % % % % % % % % %
\section{Problem 3}


% % % % % % % % % % % % % % % % % % % % % % % % % % % % % % % % % %
% % % % % % % % % % % % % % % % % % % % % % % % % % % % % % % % % %
% % % % % % % % % % % % % % % % % % % % % % % % % % % % % % % % % %
\section{Problem 4}

\begin{algorithm}
\caption{QuickMiss(C, D, start, end)}\label{QuickMiss}
    \begin{algorithmic}[1]
        \Procedure{QuickMiss}{C, D, start, end}
        % \State $l \gets m$
        % \State $p \gets n$
        \If {$p < r$}
            \State $left, q, right \gets \textsc{Partition}(C, D, start, end)$
            \If {$left < right$}
                \State $\textsc{Partition}(C, q+1, r)$
            \ElsIf {$right > left$}
                \State $\textsc{Partition}(C, p, q-1)$
            \Else
                \State \Return $D[q]$
            \EndIf
        \EndIf
        \EndProcedure
    \end{algorithmic}
\end{algorithm}


\begin{algorithm}
\caption{Partition(C, D, p, r)}\label{Partition}
    \begin{algorithmic}[1]
        \Procedure{Partition}{C, D, p, r}
        \State $r \gets r+1$
        \State $C[r] \gets D[\frac{p+r}{2}]$
        \State $q \gets p$
        \For {$i \gets p$ \textbf{to} $r-1$}
            \If {$\textsc{Compare-Strings}(C[i], C[r])$}
                \State $\textsc{Swap}(C[i], C[q])$
                \State $q \gets q+1$
            \EndIf
        \EndFor
        \State $\textsc{Swap}(C[q], C[r])$
        \State \Return $q$
        \EndProcedure
    \end{algorithmic}
\end{algorithm}

% \section{References}
%
% \nocite{*}
% \raggedright
% \bibliography{references.bib}
% \bibliographystyle{plain}


\end{document}