\documentclass[12pt]{article}
\usepackage{setspace}
\setstretch{1.1}
\usepackage{amsmath,amssymb, amsthm}
\usepackage{graphicx}
\usepackage{bm}
\usepackage[hang, flushmargin]{footmisc}
\usepackage[colorlinks=true]{hyperref}
\usepackage[nameinlink]{cleveref}
\usepackage{footnotebackref}
\usepackage{url}
\usepackage{listings}
\usepackage[most]{tcolorbox}
\usepackage{inconsolata}
\usepackage[papersize={8.5in,11in}, margin=1in]{geometry}
\usepackage{float}
\usepackage{caption}
\usepackage{esint}
\usepackage{url}
\usepackage{enumitem}
\usepackage{subfig}
\usepackage{wasysym}
\newcommand{\inlinecode}{\texttt}
\usepackage{etoolbox}
\usepackage{algorithm}
% \usepackage{algorithmic}
\usepackage[noend]{algpseudocode}
\patchcmd{\thebibliography}{\section*{\refname}}{}{}{}


\begin{document}



\title{\textbf{EECS 340: Assignment 1}}

\author{Shaochen (Henry) ZHONG, \inlinecode{sxz517} \\ Zhitao (Robert) CHEN, \inlinecode{zxc325}}
\date{Due on 01/27/2020, submitted on 01/26/2020 \\ EECS 340, Dr. Koyuturk}
\maketitle

\section{Problem 1}
\subsection{(a)}
Without loss of generality, assume $x > y$ for $x, y \in \mathbb{Z^+}$. The loop invariant is:

\begin{gather}
    Euclidean(x, y) = Euclidean(x - y, y)
\end{gather}

\subsection{(b)}
Without loss of generality, assume $x > y$ for $x, y \in \mathbb{Z^+}$.

Let $d$ for $d \in \mathbb{Z^+}$ being the greatest common divisor of $x$ and $y$, a.k.a $d = gcd(x, y)$. As $d$ being a divisor of both $x$ and $y$, we may therefore have $x = kd$ and $y = jd$ for $k, j \in \mathbb{Z^+}$. Then we may infer:
\begin{gather}
    x - y = dk - dj = d(k - j)
\end{gather}

From \textit{Equation 2} and the known fact that $y = jd$, we may say that $d$ is also a common divisor of $x - y$ and $y$. By the defination of $gcd$, this means the upper bond of $d$ cannot be greater than $gcd(x-y, y)$. Thus we may conclude:
\begin{gather}
    gcd(x, y) = d \leq gcd(x-y, y) \nonumber \\
    \Rightarrow gcd(x, y) \leq gcd(x-y, y)
\end{gather}

Now similarly, Let $e$ for $e \in \mathbb{Z^+}$ being the greatest common divisor of $x-y$ and $y$. We may therefore have $x - y = le$ and $y = me$ for $l, m \in \mathbb{Z^+}$. Then we may infer:
\begin{gather}
    x = (x - y) + y = le + me = e(l + m)
\end{gather}

From \textit{Equation 4} and the known fact that $y = me$, we may say that $e$ is also a common divisor of $x$ and $y$. By the defination of $gcd$, this means the upper bond of $e$ cannot be greater than $gcd(x, y)$. Thus we may conclude:
\begin{gather}
    gcd(x-y, y) = e \leq gcd(x, y) \nonumber  \\
    \Rightarrow gcd(x-y, y) \leq gcd(x, y)
\end{gather}

By observing \textit{Equation 3} and \textit{Equation 5}, we may have a constraint of $gcd(x, y) = gcd(x-y, y)$. As the given method $Euclidean()$ is a $gcd$ finder, we may promote such constraint to $Euclidean(x, y) = Euclidean(x - y, y)$ due to the equivalency of $Euclidean(a, b)$ and $gcd(a, b)$.

\subsection{(c)}

The \inlinecode{while} loop always terminates as the condition $x = y$ will eventually be reached. Due to the fact that we have $x$ and $y$ for $x, y \in \mathbb{Z^+}$; without loss of generality, we assume $x > y$, therefore we must have $x' = x - y$ for $x' \in \mathbb{Z^+}$.

As we have only a finte amount of $x'$, $x''$, $x'''$... to decrease for $x', x'', \text{and}\ x''' \in \mathbb{Z^+}$, the decremental calculation of $x_{k+1} = x_{k} - y_{k}$\footnote{for $k$ being the numbers of iteration went through in the \inlinecode{while} loop.} will eventually reaches a condition where $x = y$ due to the ``well ordering'' nature of the natural numbers. Thus, the \inlinecode{while} loop always terminates.

\subsection{(d)}

In \textbf{Section 1.2}, we have proven that $gcd(x, y) = gcd(x-y, y) \Rightarrow Euclidean(x, y) = Euclidean(x - y, y)$ assuming $x > y$ for $x, y \in \mathbb{Z^+}$. Following the principle of induction, we can generalize it as:

\begin{gather}
    Euclidean(x, y) = Euclidean(x_{k+1}, y_{k+1}) \\
    \text{for $x', y' \in \mathbb{Z^+}$ while } \nonumber \\
    \text{$x_{k+1} = x_{k} - y_{k}$ if $x_{k} > y_{k}$ and $y_{k+1} = y_{k} - x_{k}$ if $y_{k} > x_{k}$} \nonumber
\end{gather}

Where $Euclidean(x_{k+1}, y_{k+1})$ is the greatest common divisor of $(x, y)$.\newline


As we haven proven in \textbf{Section 1.2}, that the \inlinecode{while} loop within the $Euclidean()$ method must terminate. Combined such finding with \textit{Equation 6}, we must have a:
\begin{gather}
    Euclidean(a, b) = Euclidean(a_{k+1}, b_{k+1}) = Euclidean(a_{k+2}, a_{k+2})
    % without loss of generality
\end{gather}

As we know by calculation that $Euclidean(a_{k+2}, a_{k+2}) = a_{k+2}$, $a_{k+2}$ will be the greatest common divisor of $Euclidean(a, b)$.

\section{Problem 2}
\subsection{(a)}

\begin{algorithm}
\caption{TwoSum(A, B, n, x) with two pointers}\label{TwoSum}
\begin{algorithmic}[1]
\Procedure{}{}
\State $j \gets n-1$
\State $i \gets 0$
\While {$i < n \textbf{ and } j \geq 0$}
    \If {\textit{$A[i] + B[j] < x$}}
        \State $i \gets i + 1$
    \ElsIf {\textit{$A[i] + B[j] > x$}}
        \State $j \gets j - 1$
    \Else
        \State \Return $i, j$
    \EndIf
\EndWhile
\State \Return $False$
\EndProcedure
\end{algorithmic}
\end{algorithm}



\subsection{(b)}
\subsection{(c)}
If $A[i] + B[j] \neq x$, then it is impossible to have any combination $(a, b)$ where \\ $a \in \{A[0], A[1], ..., A[i-1], A[i]\}$ and $b \in \{B[j], B[j+1], B[j+2], ..., B[n-1]\}$ to \\be $a + b = x$ for all \textbf{iterarted} $i, j$ within the \inlinecode{while} loop.

Assume we get $A[i] + B[j] = k$ within an interation of the \inlinecode{while} loop, we either have $k < x$ or $k > x$ (assume $A[i] + B[j] \neq x$ because otherwise the loop will be terminated and the problem will be solved). Due to the sorted nature of array $A, B$, we must have:
\begin{gather}
    \underbrace{A[0] \leq A[1] \leq ... \leq A[i-1]}_{a} \leq A[i]  \ \ \ \Rightarrow a \leq A[i]\\
    \underbrace{B[n-1] \geq ... \geq B[j+2] \geq B[j+1]}_{b} \geq B[j] \ \ \ \Rightarrow b \geq B[j]
\end{gather}

Thus, we may infer the followings from the above two \textit{Equations}:
\begin{gather}
    \text{if } k < x: \ a + B[j] \leq A[i] + B[j]  \ \ \ \Rightarrow a + B[j] < x\\
    \text{if } k > x: \ A[i] + b \geq A[i] + B[j]  \ \ \ \Rightarrow A[i] + b > x
\end{gather}

It is a bit unintuitive about the case of $a' + b'$ for $a' \in \{A[0], A[1], ..., A[i-1]\}$ and $b' \in \{B[j+1], B[j+2], ..., B[n-1]\}$. But as the value of index $i$ ($j$) travels in an incremental (decremental) fashion in the bond of $\matthbb{Z+} \in [0, n)$; the index of $a'$ must be smaller than $i$, and the index of $b'$ must be greater than $j$. Thus, if there is any $a' + b' = x$, such loop would have been terminated immediately and we should not be able to reach to indexes $i, j$ in the \inlinecode{while} loop. Therefore, as we are now at the $i, j$ iteration of the \inlinecode{while} loop, this suggests there is no $(a', b')$ combination where $a' + b' = x$.

This together with \textit{Equation 10, 11}, we have proven the loop invariant as there will be no combination of $(a, b)$ where $a \in \{A[0], A[1], ..., A[i-1], A[i]\}$ and $b \in \{B[j], B[j+1], B[j+2], ..., B[n-1]\}$ to be $a + b = x$ for all \textbf{iterarted} $i, j$ within the \inlinecode{while} loop.

\subsection{(d)}
The worst case of this algorithm is when it returns \inlinecode{False}. In such case we wil have $i$ and $j$ travel the entire bond of $\matthbb{Z+} \in [0, n)$. The loop will execute $2n$ times, thus the run time is $O(n)$.

\section{Problem 3}


% \section{References}
%
% \nocite{*}
% \raggedright
% \bibliography{references.bib}
% \bibliographystyle{plain}


\end{document}