\documentclass[11pt]{article}
\usepackage{setspace}
\setstretch{1}
\usepackage[T1]{fontenc}
\usepackage{amsmath,amssymb, amsthm}
\usepackage{graphicx}
\usepackage{bm}
\usepackage[hang, flushmargin]{footmisc}
\usepackage[colorlinks=true]{hyperref}
\usepackage[nameinlink]{cleveref}
\usepackage{footnotebackref}
\usepackage{url}
\usepackage{listings}
\usepackage[most]{tcolorbox}
\usepackage{inconsolata}
\usepackage[papersize={8.5in,11in}, margin=1in]{geometry}
\usepackage{float}
\usepackage{caption}
\usepackage{esint}
\usepackage{url}
\usepackage{enumitem}
\usepackage{subfig}
\usepackage{wasysym}
\newcommand{\ilc}{\texttt}
\usepackage{etoolbox}
\usepackage{algorithm}
% \usepackage{algorithmic}
\usepackage[noend]{algpseudocode}
\usepackage{tikz}
\usetikzlibrary{matrix,positioning,arrows.meta,arrows}
\patchcmd{\thebibliography}{\section*{\refname}}{}{}{}

\makeatletter
\renewcommand{\@seccntformat}[1]{}
\makeatother


\begin{document}



\title{\textbf{EECS 340: Assignment 5}}

\author{Shaochen (Henry) ZHONG, \ilc{sxz517}}
\date{Due and submitted on 04/06/2020 \\ EECS 340, Dr. Koyut{\"u}rk}
\maketitle

% % % % % % % % % % % % % % % % % % % % % % % % % % % % % % % % % %
% % % % % % % % % % % % % % % % % % % % % % % % % % % % % % % % % %
% % % % % % % % % % % % % % % % % % % % % % % % % % % % % % % % % %
\section{Problem 1}

Hi TA! I did \textbf{(ii)} and \textbf{(vi)} on \textbf{Problem 1(a)}. Thanks.
% % % % % % % % % % % % % % % % % % % % % % % % % % % % % % % % % %
% % % % % % % % % % % % % % % % % % % % % % % % % % % % % % % % % %
\subsection{(a)}

% % % % % % % % % % % % % % % % % % % % % % % % % % % % % % % % % %
\paragraph{(i)}

The last operation base on operator $o_{\text{last}}$ for $o_{\text{last}} \in \{o_1, o_2, o_3, ..., o_{n-1}\}$.

\paragraph{(ii)} Assuming the last operation will be $T[i, j] \ o_{\text{last}} \ T[j+1, k] = T_{\text{final}}$, where $T_{\text{final}}$ provides the optimal output for $x_1 o_1 x_2 o_2 ... x_{n-1} o_{n-1} x_n$. There must be:


\begin{itemize}
    \item $T[i, j]$ provides the optimal output for $x_i o_i x_{i+1} o_{i+1} ... x_{j}$.
    \item $T[j+1, k]$ provides the optimal output for $x_{j+1} o_{j+1} x_{j+2} o_{j+2} ... x_{k}$.
\end{itemize}

\begin{proof}
    Without lost of generality, assume $T[i, j]$ is not optimal for $x_i o_i x_{i+1} o_{i+1} ... x_{j}$, there must be an $T'[i, j]$ where $T'[i, j] > T[i,j]$. Then we may have $T'[i, j] \ o_{\text{last}} \ T[j+1, k] > T_{\text{final}}$. \\
    However, it is known that $T[i, j]$ provides the optimal output for $x_i o_i x_{i+1} o_{i+1} ... x_{j}$, for $T'[i, j] > T[i,j]$ to exist,  $T[i, j]$ is no longer optimal. Thus we may say $T[i, j]$ is optimal by contradiction.
\end{proof}

\paragraph{(iii)}

\begin{itemize}
    \item Subproblem: $T[i, j]$ for the optimal output of $x_i o_i x_{i+1} o_{i+1} ... x_{j}$.
    \item Overall problem: $T[1, n]$ for the optimal output of $x_1 o_1 x_2 o_2 ... x_{n-1} o_{n-1} x_n$.
\end{itemize}

\paragraph{(iv)}

\begin{equation}
    T[i,k] =
    \begin{cases}
        x_i & \text{if \ } i = k \\
        max\{T[i,j] \ o_j \ T[j+1, k]\} &\text{for \ } i \leq j \leq k \\
    \end{cases}\nonumber
\end{equation}

\paragraph{(v)}

We have $T(n) = \Theta(n^3)$. As there are $\Theta(n^2)$ subproblems and for each subproblem of $T[i,k]$, there are $\Theta(k-i)$ choices (which is essential $\Theta(n)$). Also each choice takes $\Theta(1)$ too execute. \\
$\Longrightarrow T(n) =  \Theta(n^2) \cdot  \Theta(n) \cdot \Theta(1) =  \Theta(n^3)$


\paragraph{(vi)}
For $X$ being a list of operands \ilc{[x\_{1}, x\_{2}, x\_{3}, ... , x\_{n}]}, likewise $O$ being a list of operators \ilc{[o\_{1}, o\_{2}, o\_{3},... , o\_{n-1}]}.

(Discussed with Yuhui ZHANG on this)

\begin{algorithm}[H]
\caption{Largest-Output(X, O, n)}
    \begin{algorithmic}[1]
        \State \text{Create two empty $n \times n$ 2D arrays: $E$ and $S$.}
        \Procedure{Largest-Output}{X, O, n}
        \For {$i \gets 1$ \textbf{to} $n$}
            \State $E[i, i] \gets X[i]$ \Comment{Get operands on diagonal}
        \EndFor
        \For {$cur \gets 2$ \textbf{to} $n$}
            \For {$i \gets 1$ \textbf{to} $n-cur+1$}
                \State {$j = i + cur - 1$}
                \State $E[i, j] \gets - \infty$
                \For {$k \gets i$ \textbf{to} $j-1$}
                    \State $holder \gets eval("E[i,k] + O[k] + E[k + 1, j]$
                    \If {$holder > E[i, j]$}
                        \State{$E[i, j] = holder$}
                        \State{$S[i, j] = k$}
                    \EndIf
                \EndFor
            \EndFor
        \EndFor
    \State Return  E, S
    \end{algorithmic}
\end{algorithm}

\begin{algorithm}[H]
\caption{Display(X, O, s, i, j)}
    \begin{algorithmic}[1]
        \Procedure{Display}{X, O, s, i, j}
        \If {$i == j$}
            \State $print \ X[i]$
        \Else
            \State $print $ \textxc{Display(X, O,s, i, s[i,j] )}
            \State $print\  O[s[i, j]]$
            \State $print $ \textxc{Display(X, O,s,  s[i,j]+1, j )}
        \EndIf

    \end{algorithmic}
\end{algorithm}



% % % % % % % % % % % % % % % % % % % % % % % % % % % % % % % % % %
% % % % % % % % % % % % % % % % % % % % % % % % % % % % % % % % % %
\subsection{(b)}


\paragraph{(i)} Determine the last coin $c_{\text{last}}$ which will be added to the change set $\{c_1, c_2, c_3, ..., c_{\text{last}-1}\}$. Where the sum of elements of such set plus $c_{\text{last}}$ should be equal to $t$.


\paragraph{(iii)}

\begin{itemize}
    \item Subproblem: $f(v)$ provides the minimum number of coins to make change for $v$, given $n$ types of coin denominations are available. There must be  $v \in [0, t]$ and $v$ must be equal to the sum of some coins from the given $n$ types of coin denominations, to represent a value that is achievable by given coins.
    \item Overall problem: $f(t)$ provides the minimum number of coins to make change for $t$, given $n$ types of coin denominations are available.
\end{itemize}


\paragraph{(iv)}

Let $n_c$ be a type of coin from the provided $n$ types of coins, which means $n_c$ can potentially hold $n$ types of values. We will try all $n_c$(s) as we don't know which coin to use for the $c_1$.


\begin{equation}
    f(v) =
    \begin{cases}
        0 & \text{if \ } v = 0 \\
        min\{f(v - n_c) + 1\} & \text{if \ } v > 0 \\
    \end{cases}\nonumber
\end{equation}


\paragraph{(v)}

We have $T(n) = \Theta(nt)$. There are $\Theta(t)$ subproblems, as we can divide at most $t$ groups of coin(s) for a total value of $t$ (as for all coins being value $1$), and for each group there are at most $n$ possible choices to iterate (thus $\Theta(n)$). Each choice takes $\Theta(1)$ to execute. \\
$\Longrightarrow T(n) =  \Theta(t) \cdot \Theta(n) \cdot \Theta(1) =  \Theta(nt)$

% % % % % % % % % % % % % % % % % % % % % % % % % % % % % % % % % %
% % % % % % % % % % % % % % % % % % % % % % % % % % % % % % % % % %
\subsection{(c)}

\paragraph{(i)} Determine what edit to make (or not making any edit) when reached the last character of $X$.

\paragraph{(iii)}
Let $X_k = x_1x_2...x_k$ and likewise for $Y_k$.

\begin{itemize}
    \item Subproblem: $D(i, j)$ provides the shortest edit distance between $X_i$ and $Y_j$.
    \item Overall problem: $D(m, n)$ provides the shortest edit distance between $X_m$ and $Y_n$.
\end{itemize}

\paragraph{(iv)}

\begin{equation}
    D(i, j) =
    \begin{cases}
        j & \text{if \ } i = 0 \\
        i & \text{if \ } j = 0 \\
        D(i-1, j-1) & \text{if \ } x_i = y_j \\
        min\{D(i, j-1) + 1, D(i-1, j) + 1, D(i-1, j-1) + 1  \} & \text{otherwise}\\
    \end{cases}\nonumber
\end{equation}

\paragraph{(v)}
We have $T(n) = \Theta(mn)$. A matrix of $m \cdot n$ is filled when two strings are listing their possible reductions of chars, thus there are $\Theta(mn)$ subproblems. Each subproblem takes $\Theta(1)$ to execute as all edit options cost the same.\\
$\Longrightarrow T(n) =  \Theta(mn) \cdot \Theta(1) =  \Theta(mn)$

% % % % % % % % % % % % % % % % % % % % % % % % % % % % % % % % % %
% % % % % % % % % % % % % % % % % % % % % % % % % % % % % % % % % %
% % % % % % % % % % % % % % % % % % % % % % % % % % % % % % % % % %
\section{Problem 2}

\subsection{(a)}
\begin{proof}
Suppose we have $r_{ij}$ for the optimal exchange between currencies $i \to j$, with an substructure of $r_{ij} = r_{ik} \cdot r_{kj}$ (for $k \in [i,j]$). We want to show both $r_{ik}$ and $r_{kj}$ must be the optimal exchange between currencies $i \to k$ and $k \to j$.

Without lost of generality, assume we have $r'_{ik}$ where $r'_{ik} > r_{ik}$, then there must be $r'_{ik} \cdot r_{kj} > r_{ij}$. Which suggest $r_{ij}$ is no longer optimal for the exchange between currencies $i \to j$ and the assumption is void. Thus, we proved substructure $r_{ik}$ must be at its optimal structure by contradiction. And an optimal $r_{ij}$ is only obtainable with optimal substcutures.
\end{proof}

\subsection{(b)}

It is certainly possible. Assume the optimal exchange path found by the abovementioned procedure is a long path with a tiny amount of gain after each exchange, e.g., for $r_{az}$ we have to do $ r_{ab} \cdot r_{bc} \cdot ... \cdot r_{yz}$, where after each exchange the accumlated value become $1.0001$ times of before, and an direct exchange rate between currencies $a \to z$ is $1.0001$ as well.

Supposely the best exchange path with no exchange fee is to do as many of positive exchanges as possible, as the accumlated values will only be larger and larger, result in a maximum amount of currency $z$ at the end. But if we set $f(k) = 0.5$ for all exchanges, the more exchange we do, the more money we lost -- since the tiny amount of gain of one exchange cannot cover the high exchange fee. In this case, the best exchange path would be a direct exchange between currencies $a \to z$ is $1$.

Since the optimal exchange path with $f(k) = 0.5$ is not the same as the optimal path produced by the abovementioned procedure, we have proven the statement.

% \section{References}
%
% \nocite{*}
% \raggedright
% \bibliography{references.bib}
% \bibliographystyle{plain}


\end{document}