\documentclass[12pt]{article}
\usepackage{setspace}
\setstretch{1}
\usepackage{amsmath,amssymb, amsthm}
\usepackage{graphicx}
\usepackage{bm}
\usepackage[hang, flushmargin]{footmisc}
\usepackage[colorlinks=true]{hyperref}
\usepackage[nameinlink]{cleveref}
\usepackage{footnotebackref}
\usepackage{url}
\usepackage{listings}
\usepackage[most]{tcolorbox}
\usepackage{inconsolata}
\usepackage[papersize={8.5in,11in}, margin=1in]{geometry}
\usepackage{float}
\usepackage{caption}
\usepackage{esint}
\usepackage{url}
\usepackage{enumitem}
\usepackage{subfig}
\usepackage{wasysym}
\newcommand{\inlinecode}{\texttt}
\usepackage{etoolbox}
\usepackage{algorithm}
% \usepackage{algorithmic}
\usepackage[noend]{algpseudocode}
\usepackage{tikz}
\usetikzlibrary{matrix,positioning,arrows.meta,arrows}
\patchcmd{\thebibliography}{\section*{\refname}}{}{}{}

\makeatletter
\renewcommand{\@seccntformat}[1]{}
\makeatother


\begin{document}



\title{\textbf{EECS 340: Assignment 2}}

\author{Shaochen (Henry) ZHONG, \inlinecode{sxz517} \\ Zhitao (Robert) CHEN, \inlinecode{zxc325}}
\date{Due and submitted on 02/03/2020 \\ EECS 340, Dr. Koyutürk}
\maketitle

\section{Problem 1}

% % % % % % % % % % % % % % % % % % % % % % % % % % % % % % % % % %
% % % % % % % % % % % % % % % % % % % % % % % % % % % % % % % % % %
% % % % % % % % % % % % % % % % % % % % % % % % % % % % % % % % % %
\subsection{(a) $max\{f(n), g(n)\} = \Theta (f(n) + g(n))$}

Since it is known that $f(n) \geq 0$, $g(n) \geq 0$, and $c > 0$; we must have:

\begin{gather}
    f(n) \leq f(n) + g(n) \nonumber\\
    g(n) \leq f(n) + g(n) \nonumber\\
    \Rightarrow \max(f(n), g(n)) \in O(f(n) + g(n)) \ \ \ \text{for } \begin{cases}
                    c = 1 \\
                    \forall{n_{0}} \in \mathbb{R}
                \end{cases}
\end{gather}

Since it is also known that $f(n) + g(n) \leq 2 \cdot \max(f(n),g(n))$, we may therefore infer:

\begin{gather}
    \max(f(n), g(n)) \in \Omega(f(n) + g(n)) \ \ \ \text{for } \begin{cases}
        c = \frac{1}{2} \\
        \forall{n_{0}} \in \mathbb{R}
    \end{cases}
\end{gather}

Since both the $O$- and $\Omega$-notation are established, we may therefore conclude:

\begin{gather}
    \max(f(n), g(n)) \in \Theta (f(n) + g(n))
\end{gather}


% % % % % % % % % % % % % % % % % % % % % % % % % % % % % % % % % %
% % % % % % % % % % % % % % % % % % % % % % % % % % % % % % % % % %
% % % % % % % % % % % % % % % % % % % % % % % % % % % % % % % % % %
\subsection{(b1) $f(n) + d = O(f(n))$.}

\textbf{False}

For the seek of disambiguation, we rewrite the questioned equation as $f(n) + d = O(f(n))$ by $d$ replacing $c$ for $d > 0$.

To prove the statement to be valid, we need to show:

\begin{gather}
    0 \leq f(n) + d \leq cf(n)
\end{gather}


For $f(n) = 0$, we cannot find any $c$ which satisfies the above equation since:

\begin{gather}
    O(f(n)) = cf(n) = 0  \nonumber \\
    \Rightarrow 0 \leq 0 + d \leq c(0) \ \ \ \text{for $d > 0$}
\end{gather}

As the equation $0 + d \leq 0$ leads to a contradiction, the statement is invalid.


% % % % % % % % % % % % % % % % % % % % % % % % % % % % % % % % % %
% % % % % % % % % % % % % % % % % % % % % % % % % % % % % % % % % %
% % % % % % % % % % % % % % % % % % % % % % % % % % % % % % % % % %
\subsection{(b2) If $f(n) \geq 1$, then $f(n) + d = O(f(n))$.}

For the seek of disambiguation, we rewrite the questioned equation as $f(n) + d = O(f(n))$ by $d$ replacing $c$ for $d > 0$.


According to the defination of $O$-notation, we have:

\begin{gather}
 f(n) = O(f(n)) \nonumber\\
 \exists c, n_{0} > 0 \ \text{ s.t.  $0 \leq f(n) \leq cf(n)$ \ \ \ for $n \geq n_{0}$} \\
 \exists n' > 0 \ \text{ s.t.  $f(n) \geq f(n')$ \ \ \ for $n, n' \geq n_{0}$} \\
 \Rightarrow 0 \leq f(n)+d \leq cf(n) + d \ \ \ \text{for $n \geq n_{0}$}
\end{gather}

Thus, we may rewrite the above equation as:


\begin{gather}
    0 \leq f(n)+d \leq (c + \frac{d}{f(n)})f(n) \ \ \ \text{for $f(n) \geq 1$} \nonmumber\\
    \Rightarrow 0 \leq f(n)+d \ \leq \ c'f(n) \ \ \ \text{for } \begin{cases}
                     c' = c + \frac{d}{f(n')} \\
                     n, n' \geq n_{0}
                 \end{cases}
\end{gather}

Therefore, we may conclude if $f(n) \geq 1$, then $f(n) + d = O(f(n))$ via direct proof.




% % % % % % % % % % % % % % % % % % % % % % % % % % % % % % % % % %
% % % % % % % % % % % % % % % % % % % % % % % % % % % % % % % % % %
% % % % % % % % % % % % % % % % % % % % % % % % % % % % % % % % % %
\subsection{(c1) If $f(n) = O(g(n)), \log(f(n)) \geq 0$ and $\log(g(n)) \geq 0$, then $log(f(n)) = O(\log(g(n))$.}

\textbf{False}

For $f(n) = 2$ and $g(n) = 1$, then $\log f(n) = 1$ and $\log g(n) = 0$. Since we cannot find any constant $c$ where:

\begin{gather}
    0 \leq 1 \leq c(0)
\end{gather}

Thus the statement is invalid.

% % % % % % % % % % % % % % % % % % % % % % % % % % % % % % % % % %
% % % % % % % % % % % % % % % % % % % % % % % % % % % % % % % % % %
% % % % % % % % % % % % % % % % % % % % % % % % % % % % % % % % % %
\subsection{(c2) If $f(n) = O(g(n)), \log(f(n)) \geq 0$ and $\log(g(n)) \geq 1$, then $log(f(n)) = O(\log(g(n))$.}



According to the defination of $O$-notation, we must have:

\begin{gather}
    \exists c, n_{0} > 0 \ \text{ s.t. } \ f(n) \leq cg(n) \ \ \ \text{for $n \geq n_{0}$}
\end{gather}


Since it is given that $\log(f(n))\geq 0$, $\log(g(n)) \geq 1 $, thus we must have:

\begin{gather}
    \log(f(n)) \leq \log(c(g(n))) \ \ \ \text{for $n \geq n_{0}$} \nonumber\\
    \Rightarrow \log(f(n)) \leq \log c + \log(g(n)) \ \ \ \text{for $n \geq n_{0}$}
\end{gather}

As $c, n_{0}$ are constants, there must be a constant $c'$ s.t.


\begin{gather}
    c' \geq \frac{\log c}{\log (g(n_{0}))} + 1 \\
    \Rightarrow (c' -1)\log (g(n))  \ \geq \ (c' -1)\log (g(n_{0})) \ \geq \ \log c \ \ \ \text{for $n \geq n_{0}$} \\
    \exists c, n_{0} > 0 \nonumber \ \text{ s.t. } \  \\
    \log(f(n)) \ \leq \ \log c + \log(g(n)) \ \leq \  (c' -1)\log (g(n)) + \log(g(n))\ \ \ \text{for $n \geq n_{0}$} \\
    \Rightarrow \log(f(n))  \leq c'\log (g(n))
\end{gather}

Thus we may conclude $log(f(n)) = O(\log(g(n))$, the statement is therefore proven to be valid.


% % % % % % % % % % % % % % % % % % % % % % % % % % % % % % % % % %
% % % % % % % % % % % % % % % % % % % % % % % % % % % % % % % % % %
% % % % % % % % % % % % % % % % % % % % % % % % % % % % % % % % % %
\subsection{(d1) $f(2n) = \Theta(f(n))$}

\textbf{False}

For $f(n) = 2^n$, we have $f(2n) = 4^n$. Where $f(2n) \neq \Theta(f(n))$ due to the LHS has a higher asymptotic order, and therefore we can't find any constant $c_{1}, c_{2}$ to form a relation of $c_1 \cdot 2^n \leq 4^n \leq c_2 \cdot 2^n$. Thus, the statement is invalid.


% % % % % % % % % % % % % % % % % % % % % % % % % % % % % % % % % %
% % % % % % % % % % % % % % % % % % % % % % % % % % % % % % % % % %
% % % % % % % % % % % % % % % % % % % % % % % % % % % % % % % % % %
\subsection{(d2) If $f(n) = O(n^k)$, then $f(2n) = O(n^k)$}


For the seek of disambiguation, we rewrite the questioned equation as: if $f(n) = O(n^k)$, then $f(2n) = O(n^k)$ by $k$ replacing $c$ for $k > 0$.


Since $f(n) = O(n^k)$, we must have $0 \leq f(n) \leq cn^k$ for $n \geq n_0$ and a $c$ for $c \in \mathbb{R^+}$. Now substitute $n$ as $2n$, we may have:

\begin{gather}
    0 \leq f(2n) \leq c(2n)^k \nonumber \\
    \Rightarrow 0 \leq f(2n) \leq c\cdot (2)^k \cdot n^k \\
    \Rightarrow 0 \leq f(2n) \leq c' \cdot n^k \ \ \ \text{where $c' = c\cdot (2)^k$}
\end{gather}

Thus we may conclude if $f(n) = O(n^k)$, then $f(2n) = O(n^k)$.

% % % % % % % % % % % % % % % % % % % % % % % % % % % % % % % % % %
% % % % % % % % % % % % % % % % % % % % % % % % % % % % % % % % % %
% % % % % % % % % % % % % % % % % % % % % % % % % % % % % % % % % %
\subsection{(d3) If $f(n) = \Theta(n^k)$, then $f(2n) = \Theta(f(n))$}

For the seek of disambiguation, we rewrite the questioned equation as: if $f(n) = \Theta(n^k)$, then $f(2n) = \Theta(f(n))$.

Since $f(n) = \Theta(n^k)$, we must have $0 \leq c_1 \cdot n^k \leq f(n) \leq c_2 \cdot n^k$ for $n \geq n_0$ and $c_1, c_2$ for $c_1, c_2 \in \mathbb{R^+}$. Now substitute $n$ as $2n$, we may have:

\begin{gather}
    0 \leq c_1 (2n)^k \leq f(2n) \leq c_2 (2n)^k \nonumber \\
    \Rightarrow 0 \leq c_1 \cdot 2^k \cdot n^k \leq f(2n) \leq c_2 \cdot 2^k \cdot n^k
\end{gather}

From $0 \leq c_1 \cdot n^k \leq f(n) \leq c_2 \cdot n^k$ , we may also infer:

\begin{gather}
    \frac{f(n)}{c_1} \geq k \geq \frac{f(n)}{c_2} \\
    \Rightarrow c_1 \cdot 2^k \cdot \frac{f(n)}{c_2} \leq f(2n) \leq c_2 \cdot 2^k \cdot \frac{f(n)}{c_1} \\
    \Rightarrow 0\leq c^{'}_{1}\cdot f(n) \leq f(2n) \leq c^{'}_{2}\cdot f(2n) \ \ \ \text{for}\begin{cases}
                    c^{'}_{1} = \frac{c_1 \cdot 2^k}{c_2} \\
                    c^{'}_{2} = \frac{c_2 \cdot 2^k}{c_1}
                \end{cases}
\end{gather}


Thus we may conclude if $f(n) = \Theta(n^k)$, then $f(2n) = \Theta(f(n))$.

\section{Problem 2}

\paragraph{Answer}

\begin{gather}
     n^{-a} \ll n^{-\epsilon} \ll \epsilon^n  \\
    \ll \log(n^\epsilon) \equiv \log(bn) \equiv \log(n^a) \equiv \log(n^b) \equiv \log_{\frac{1}{\epsilon}}(n)\\
    \ll  (\log n)^a \ll n^{\epsilon} \ll a^{\log_a(n)} \\
    \equiv \epsilon n \equiv \frac{n}{a} \\
    \ll n^a \equiv (n+b)^a \ll (n+a)^b \ll n^{a+b} \\
    \ll a^{\epsilon n} \equiv a^n \ll b^n
\end{gather}


\paragraph{Justification of $\log(n^\epsilon) \equiv \log(bn) \equiv \log(n^a) \equiv \log(n^b)$\newline}

The above equations can be rewrite as:

\begin{gather}
    \log(n^\epsilon) = \epsilon \log(n) \\
    \log(n^a) = a \log(n) \\
    \log(bn) = \log(b) + \log(n) \\
    \log(n^b) = b \log(n)
\end{gather}

Where all of them can be generalized as $\Theta(\log(n))$, as it is known that $n^\epsilon \leq bn \leq n^a$ due to the decending of \textit{power(s)}.

\paragraph{Justification of $ \log_{\frac{1}{\epsilon}}(n) \text{ \ (when $0 < \epsilon < \frac{1}{2}$)} \equiv \log(n^b) \leq \log_{\frac{1}{\epsilon}}(n) \text{ \ (when $\epsilon \geq \frac{1}{2}$)}$\newline}


It is known that when $\epsilon < \frac{1}{2} \Rightarrow \frac{1}{\epsilon} \geq 2$; when $\epsilon \geq \frac{1}{2} \Rightarrow  \frac{1}{\epsilon} \leq 2$. Since it is observable that $\log_{x}(n) \gg \log_{x'}(n)$ for $x < x'$, such (in)equaility is valid.


\paragraph{Justification of $a^{\log_a(n)} \equiv \epsilon n \equiv \frac{n}{a}$\newline}

The above equations can be rewrite as:
\begin{gather}
    a^{\log_a(n)} = n^1 = \Theta(n) \\
    \epsilon n = \Theta(n) \\
    frac{n}{a} = \frac{1}{a}n =  \Theta(n)
\end{gather}

Thus, the above equality is justified.

\paragraph{Justification of $a^{\epsilon n} \equiv a^n$\newline}

We may rewrite $a^{\epsilon n}$ as the following, since there must be a $c$ for $c \in \mathbb{R^+}$ which satisfy the equality.

\begin{gather}
    a^{\epsilon} \cdot a^n = c \cdot a^n = \Theta(a^n)
\end{gather}

Since it is known that $a^n = \Theta(a^n)$, these two expressions are considered equivalent.

% \section{References}
%
% \nocite{*}
% \raggedright
% \bibliography{references.bib}
% \bibliographystyle{plain}


\end{document}