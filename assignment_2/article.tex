\documentclass[12pt]{article}
\usepackage{setspace}
\setstretch{1}
\usepackage{amsmath,amssymb, amsthm}
\usepackage{graphicx}
\usepackage{bm}
\usepackage[hang, flushmargin]{footmisc}
\usepackage[colorlinks=true]{hyperref}
\usepackage[nameinlink]{cleveref}
\usepackage{footnotebackref}
\usepackage{url}
\usepackage{listings}
\usepackage[most]{tcolorbox}
\usepackage{inconsolata}
\usepackage[papersize={8.5in,11in}, margin=1in]{geometry}
\usepackage{float}
\usepackage{caption}
\usepackage{esint}
\usepackage{url}
\usepackage{enumitem}
\usepackage{subfig}
\usepackage{wasysym}
\newcommand{\inlinecode}{\texttt}
\usepackage{etoolbox}
\usepackage{algorithm}
% \usepackage{algorithmic}
\usepackage[noend]{algpseudocode}
\usepackage{tikz}
\usetikzlibrary{matrix,positioning,arrows.meta,arrows}
\patchcmd{\thebibliography}{\section*{\refname}}{}{}{}

\makeatletter
\renewcommand{\@seccntformat}[1]{}
\makeatother


\begin{document}



\title{\textbf{EECS 340: Assignment 2}}

\author{Shaochen (Henry) ZHONG, \inlinecode{sxz517} \\ Zhitao (Robert) CHEN, \inlinecode{zxc325}}
\date{Due and submitted on 02/03/2020 \\ EECS 340, Dr. Koyutürk}
\maketitle

\section{Problem 1}

% % % % % % % % % % % % % % % % % % % % % % % % % % % % % % % % % %
% % % % % % % % % % % % % % % % % % % % % % % % % % % % % % % % % %
% % % % % % % % % % % % % % % % % % % % % % % % % % % % % % % % % %
\subsection{(a) $max\{f(n), g(n)\} = \Theta (f(n) + g(n))$}

Since it is known that $f(n) \geq 0$, $g(n) \geq 0$, and $c > 0$; we must have:

\begin{gather}
    f(n) \leq f(n) + g(n) \nonumber\\
    g(n) \leq f(n) + g(n) \nonumber\\
    \Rightarrow \max(f(n), g(n)) \in O(f(n) + g(n)) \ \ \ \text{for } \begin{cases}
                    c = 1 \\
                    \forall{n_{0}} \in \mathbb{R}
                \end{cases}
\end{gather}

Since it is also known that $f(n) + g(n) \leq 2 \cdot \max(f(n),g(n))$, we may therefore infer:

\begin{gather}
    \max(f(n), g(n)) \in \Omega(f(n) + g(n)) \ \ \ \text{for } \begin{cases}
        c = \frac{1}{2} \\
        \forall{n_{0}} \in \mathbb{R}
    \end{cases}
\end{gather}

Since both the $O$- and $\Omega$-notation are established, we may therefore conclude:

\begin{gather}
    \max(f(n), g(n)) \in \Theta (f(n) + g(n))
\end{gather}


% % % % % % % % % % % % % % % % % % % % % % % % % % % % % % % % % %
% % % % % % % % % % % % % % % % % % % % % % % % % % % % % % % % % %
% % % % % % % % % % % % % % % % % % % % % % % % % % % % % % % % % %
\subsection{(b1) $f(n) + d = O(f(n))$.}

For the seek of disambiguation, we rewrite the questioned equation as $f(n) + d = O(f(n))$ by $d$ replacing $c$ for $d > 0$.


According to the defination of $O$-notation, we have:

\begin{gather}
    f(n) = O(f(n)) \nonumber\\
    \exists c, n_{0} > 0 \ \text{ s.t.  $0 \leq f(n) \leq cf(n)$ \ \ \ for $n \geq n_{0}$} \\
    \exists n' > 0 \ \text{ s.t.  $f(n) \geq f(n')$ \ \ \ for $n, n' \geq n_{0}$} \\
    \Rightarrow 0 \leq f(n)+d \leq cf(n) + d \ \ \ \text{for $n \geq n_{0}$}
\end{gather}

Due to \textit{Equation 5}, we may rewrite \textit{Euqation 6} as:


\begin{gather}
    0 \leq f(n)+d \leq (c + \frac{d}{f(n)})f(n) \nonmumber\\
    \Rightarrow 0 \leq f(n)+d \ \leq \ c'f(n) \ \ \ \text{for } \begin{cases}
                    c' = c + \frac{d}{f(n')} \\
                    n, n' \geq n_{0}
                \end{cases}
\end{gather}

Based \textit{Euqation 8}, we may conclude $f(n) + d = O(f(n))$ via direct proof.


% % % % % % % % % % % % % % % % % % % % % % % % % % % % % % % % % %
% % % % % % % % % % % % % % % % % % % % % % % % % % % % % % % % % %
% % % % % % % % % % % % % % % % % % % % % % % % % % % % % % % % % %
\subsection{(b2) If $f(n) \geq 1$, then $f(n) + c = O(f(n))$.}

Please refer to proof at \textbf{b1} as it provides a broader proof base on $f(n)$ regardless $f(n) \geq 1$ or not.


% % % % % % % % % % % % % % % % % % % % % % % % % % % % % % % % % %
% % % % % % % % % % % % % % % % % % % % % % % % % % % % % % % % % %
% % % % % % % % % % % % % % % % % % % % % % % % % % % % % % % % % %
\subsection{(c1) If $f(n) = O(g(n)), \log(f(n)) \geq 0$ and $\log(g(n)) \geq 0$, then $log(f(n)) = O(\log(g(n))$.}

According to the defination of $O$-notation, we must have:

\begin{gather}
    \exists c, n_{0} > 0 \ \text{ s.t. } \ f(n) \leq cg(n) \ \ \ \text{for $n \geq n_{0}$}
\end{gather}


Since it is given that $\log(f(n)), \log(g(n)) \geq 0$, thus we must have:

\begin{gather}
    \log(f(n)) \leq \log(c(g(n))) \ \ \ \text{for $n \geq n_{0}$} \nonumber\\
    \Rightarrow \log(f(n)) \leq \log c + \log(g(n)) \ \ \ \text{for $n \geq n_{0}$}
\end{gather}

As $c, n_{0}$ are constants, there must be a constant $c'$ s.t.

\paragraph{Case 1} Assume $\log (g(n_{0})) \neq 0$:

\begin{gather}
    c' \geq \frac{\log c}{\log (g(n_{0}))} + 1 \\
    \Rightarrow (c' -1)\log (g(n))  \ \geq \ (c' -1)\log (g(n_{0})) \ \geq \ \log c \ \ \ \text{for $n \geq n_{0}$} \\
    \exists c, n_{0} > 0 \nonumber \ \text{ s.t. } \  \\
    \log(f(n)) \ \leq \ \log c + \log(g(n)) \ \leq \  (c' -1)\log (g(n)) + \log(g(n))\ \ \ \text{for $n \geq n_{0}$} \\
    \Rightarrow \log(f(n))  \leq c'\log (g(n))
\end{gather}

Thus we may conclude $log(f(n)) = O(\log(g(n))$ for this case.


\paragraph{Case 2} Assume $\log (g(n_{0})) = 0$:

Since $\log (g(n_{0})) = 0$, we shall infer that $g(n_{0}) = 1$. We may arbitrarily pick some constants $c, c'}$ where:

\begin{gather}
    \log c \leq 0 \nonumber\\
    \log c \leq (c' - 1) \cdot 0 \nonumber\\
    \log c \leq (c' - 1) \cdot \log(g(n_{0}))
\end{gather}

Since $n \geq n_{0}$ by defination, and known that $g(n) \geq 1$ due to $\log(g(n)) \geq 0$; therefore there must be $g(n) \geq g(n_{0})$. Putting this into the context of \textit{Equation 10}, we may have:

\begin{gather}
    \log(f(n)) \leq \log c + \log(g(n)) \ \ \ \text{for $n \geq n_{0}$} \nonumber \\
    \Rightarrow \log(f(n)) \leq (c' - 1) \cdot \log(g(n_{0})) + \log(g(n)) \nonumber \\
    \Rightarrow \log(f(n)) \leq (c' - 1) \cdot \log(g(n)) + \log(g(n)) \nonumber \\
    \Rightarrow \log(f(n)) \leq c'\log(g(n))
\end{gather}

Thus we may conclude $log(f(n)) = O(\log(g(n))$ for this case.\newline

Since both cases reach to the conclusion of $log(f(n)) = O(\log(g(n))$, we have proven the statement to be valid.



% % % % % % % % % % % % % % % % % % % % % % % % % % % % % % % % % %
% % % % % % % % % % % % % % % % % % % % % % % % % % % % % % % % % %
% % % % % % % % % % % % % % % % % % % % % % % % % % % % % % % % % %
\subsection{(c2) If $f(n) = O(g(n)), \log(f(n)) \geq 0$ and $\log(g(n)) \geq 1$, then $log(f(n)) = O(\log(g(n))$.}



According to the defination of $O$-notation, we must have:

\begin{gather}
    \exists c, n_{0} > 0 \ \text{ s.t. } \ f(n) \leq cg(n) \ \ \ \text{for $n \geq n_{0}$}
\end{gather}


Since it is given that $\log(f(n))\geq 0$, $\log(g(n)) \geq 1 $, thus we must have:

\begin{gather}
    \log(f(n)) \leq \log(c(g(n))) \ \ \ \text{for $n \geq n_{0}$} \nonumber\\
    \Rightarrow \log(f(n)) \leq \log c + \log(g(n)) \ \ \ \text{for $n \geq n_{0}$}
\end{gather}

As $c, n_{0}$ are constants, there must be a constant $c'$ s.t.


\begin{gather}
    c' \geq \frac{\log c}{\log (g(n_{0}))} + 1 \\
    \Rightarrow (c' -1)\log (g(n))  \ \geq \ (c' -1)\log (g(n_{0})) \ \geq \ \log c \ \ \ \text{for $n \geq n_{0}$} \\
    \exists c, n_{0} > 0 \nonumber \ \text{ s.t. } \  \\
    \log(f(n)) \ \leq \ \log c + \log(g(n)) \ \leq \  (c' -1)\log (g(n)) + \log(g(n))\ \ \ \text{for $n \geq n_{0}$} \\
    \Rightarrow \log(f(n))  \leq c'\log (g(n))
\end{gather}

Thus we may conclude $log(f(n)) = O(\log(g(n))$, the statement is therefore proven to be valid.


% % % % % % % % % % % % % % % % % % % % % % % % % % % % % % % % % %
% % % % % % % % % % % % % % % % % % % % % % % % % % % % % % % % % %
% % % % % % % % % % % % % % % % % % % % % % % % % % % % % % % % % %
\subsection{(d1) $f(2n) = \Theta(f(n))$}

Since it is known that $f(n) \geq 0$, $g(n) \geq 0$, and $c > 0$; we must have a constant $c' > 0$ which satisfy:

\paragraph{Case 1} Assume $f(n) \neq 0$.


\begin{gather}
    c' \geq \frac{f(2n)}{f(n)} \\
    \Rightarrow f(2n) \leq c'(f(n))
\end{gather}

Thus we may conclude $f(2n) = O(f(n))$.

Similarly, we may also have a constant $c'' > 0$ which satisfy:


\begin{gather}
    c'' \leq \frac{f(2n)}{f(n)} \\
    \Rightarrow f(2n) \geq c''(f(n))
\end{gather}

Thus we may conclude $f(2n) = \Omega(f(n))$.\newline

Since both the $O$- and $\Omega$-notation are established, we may therefore conclude $f(2n) = \Theta(f(n))$ in this case.

\paragraph{Case 1} Assume $f(n) = 0$ for $\all n$.

Thus we must have two constants $k_{1}, k_{2}$ which satisfy:
\begin{gather}
    f(2n) = 0 \\
    k_{1}(0) \leq f(2n) \leq k_{2}f(n)\Rightarrow k_{1}(0) \leq 0 \leq k_{2}(0), \ \forall k \in \mathbb{R^+}
\end{gather}

Thus we may conclude $f(2n) = \Theta(f(n))$.\newline


Since both cases reach to the conclusion of $f(2n) = \Theta(f(n))$, we have proven the statement to be valid.


\subsection{(d2) If $f(n) = O(n^c)$, then $f(2n) = O(n^c)$}

As It is known from \textbf{d1} that $f(2n) = \Theta(f(n))$, which implies $f(2n) = O(f(n))$. Also it is given that $f(n) = O(n^c)$. Together, we shall infer $f(2n) = O(n^c)$ due to the transitivity property of $\Theta$- and $O$-notations.


\subsection{(d3) If $f(n) = \Theta(n^c)$, then $f(2n) = \Theta(f(n))$}
Please refer to proof at \textbf{d3} as it provides a broader proof base on $f(n)$ regardless $f(n) = \Theta(n^c)$ or not.


\section{Problem 2}

\paragraph{Answer}

\begin{gather}
    \frac{1}{n^a} \ll \frac{1}{n^\epsilon} \ll \log(\frac{1}{\epsilon}n) \text{ \ (when $0 < \epsilon < \frac{1}{2}$)} \ll \log(n^\epsilon) \equiv \log(bn) \equiv \log(n^a) \\
    \equiv \log(n^b) \equiv \log(\frac{1}{\epsilon}n) \text{ \ (when $\epsilon \geq \frac{1}{2}$)} \ll (\log n)^a \ll n^{\epsilon} \\
    \ll a^{log_a(n)} \equiv \epsilon n \equiv \frac{n}{a} \ll n^a \equiv (n+b)^a\\
    \ll (n+a)^b \ll n^{a+b} \ll \epsilon^n \ll a^{\epsilon n} \equiv a^n \ll b^n
\end{gather}


% \section{References}
%
% \nocite{*}
% \raggedright
% \bibliography{references.bib}
% \bibliographystyle{plain}




\end{document}