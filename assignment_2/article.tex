\documentclass[12pt]{article}
\usepackage{setspace}
\setstretch{1}
\usepackage{amsmath,amssymb, amsthm}
\usepackage{graphicx}
\usepackage{bm}
\usepackage[hang, flushmargin]{footmisc}
\usepackage[colorlinks=true]{hyperref}
\usepackage[nameinlink]{cleveref}
\usepackage{footnotebackref}
\usepackage{url}
\usepackage{listings}
\usepackage[most]{tcolorbox}
\usepackage{inconsolata}
\usepackage[papersize={8.5in,11in}, margin=1in]{geometry}
\usepackage{float}
\usepackage{caption}
\usepackage{esint}
\usepackage{url}
\usepackage{enumitem}
\usepackage{subfig}
\usepackage{wasysym}
\newcommand{\inlinecode}{\texttt}
\usepackage{etoolbox}
\usepackage{algorithm}
% \usepackage{algorithmic}
\usepackage[noend]{algpseudocode}
\usepackage{tikz}
\usetikzlibrary{matrix,positioning,arrows.meta,arrows}
\patchcmd{\thebibliography}{\section*{\refname}}{}{}{}

\makeatletter
\renewcommand{\@seccntformat}[1]{}
\makeatother


\begin{document}



\title{\textbf{EECS 340: Assignment 2}}

\author{Shaochen (Henry) ZHONG, \inlinecode{sxz517} \\ Zhitao (Robert) CHEN, \inlinecode{zxc325}}
\date{Due and submitted on 02/03/2020 \\ EECS 340, Dr. Koyutürk}
\maketitle

\section{Problem 1}

% % % % % % % % % % % % % % % % % % % % % % % % % % % % % % % % % %
% % % % % % % % % % % % % % % % % % % % % % % % % % % % % % % % % %
% % % % % % % % % % % % % % % % % % % % % % % % % % % % % % % % % %
\subsection{(a) $max\{f(n), g(n)\} = \Theta (f(n) + g(n))$}

Since it is known that $f(n) \geq 0$, $g(n) \geq 0$, and $c > 0$; we must have:

\begin{gather}
    f(n) \leq f(n) + g(n) \nonumber\\
    g(n) \leq f(n) + g(n) \nonumber\\
    \Rightarrow \max(f(n), g(n)) \in O(f(n) + g(n)) \ \ \ \text{for } \begin{cases}
                    c = 1 \\
                    \forall{n_{0}} \in \mathbb{R}
                \end{cases}
\end{gather}

Since it is also known that $f(n) + g(n) \leq 2 \cdot \max(f(n),g(n))$, we may therefore infer:

\begin{gather}
    \max(f(n), g(n)) \in \Omega(f(n) + g(n)) \ \ \ \text{for } \begin{cases}
        c = \frac{1}{2} \\
        \forall{n_{0}} \in \mathbb{R}
    \end{cases}
\end{gather}

Since both the $O$- and $\Omega$-notation are established, we may therefore conclude:

\begin{gather}
    \max(f(n), g(n)) \in \Theta (f(n) + g(n))
\end{gather}

\section{Problem 2}


% \section{References}
%
% \nocite{*}
% \raggedright
% \bibliography{references.bib}
% \bibliographystyle{plain}


\end{document}